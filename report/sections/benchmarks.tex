%!TEX root = ../Work-stealing Queues.tex
\section{Benchmarks}
To test our different queue implementations, we developed a small framework
for parallelizing computations using work-stealing queues. There are three
main Scala components: the \texttt{WorkerPool}, \texttt{Worker}, and \texttt{Node}
traits. For each of the cases below, classes implementing these traits were
created. The \texttt{WorkerPool} is in charge of starting the desired number
of threads, and assigning each \texttt{Worker} to a thread. A \texttt{Worker}
holds a queue, and contains the main logic for the case. \texttt{Node}s
represent individual ``work units'', and are the elements contained in the
queues.

\label{sec:benchmarks}
\subsection{Cases}
To test our different work-stealing queues, we have implemented four different
example cases.
\subsubsection{Raw Queue Operations} %Raw

The first of these cases, which we call ``Raw'', is meant to do as little work as possible,
and thus stress the queue implementations as much as possible. A tree of \texttt{Node}s is built, each of which hold an
arbitrary number of children, and nothing else. Each \texttt{Worker} simply
looks at a \texttt{Node}, and adds all its children to the queue, before
moving to the next \texttt{Node}. This means that queue operations will be the
deciding factor for how long a run will take. It should be noted that this is
an unrealistic workload, as no actual work is done, besides traversing the
tree.

\subsubsection{Quick Sort} % Quick sort
Our second case is a simple quick sort implementation. It is simple in that it
does not sort in-place. Instead, when splitting an array, each new child
\texttt{Node} receives a full copy of the array to split. When a lower
threshold for array length is reached, in-place insertion sort is used. Once an array
has been sorted using insertion sort, its parent \texttt{Node} is notified. If
both the parent \texttt{Node}'s children have been sorted, the parent
\texttt{Node} is added to the queue. When the parent \texttt{Node} is examined
by a \texttt{Worker}, the two sorted subarrays are combined, and the next
parent \texttt{Node} in the hierarchy is notified. When both children of the
root \texttt{Node} are combined, the entire array has been sorted. 

In Section~\ref{sec:implementations} we discussed different queue implementations, some of which only work with idempotent tasks. At first glance, quick sort should be an idempotent problem as it is fine to sort a list multiple times. The problem becomes apparent when we consider the actual implementation. Since the sorting in the leafs is done by an in-place insertion sort, inconsistencies can occur if two workers sort the same list at the same time. This could be solved by adding synchronization to the insertion sort, but that somewhat defeats the purpose of work-stealing queues.

\subsubsection{Spanning Tree} % Spanning tree
For our third case, we implemented the parallel spanning tree algorithm
described by Bader and Cong~\cite{Bader04afast}, with some small differences.
Their algorithm functions by first letting a single thread build a ``stub
tree'', a small portion of the full tree built by randomly walking the graph.
This tree's vertices are then evenly distributed into each queue. After this,
each thread starts consuming work from their queues. Our implementation is
slightly different, in that we do not build the initial stub tree. Instead, the
first \texttt{Worker} is given a node in the graph to work on, and as it
generates more work, the other \texttt{Worker}s can steal from it. This change
was purely to make the implementation simpler, and does not affect the
correctness of the algorithm, although it might make the initial steps slightly
slower.

\subsubsection{XML Serialization} % XML serialization
Our final case was inspired by Lu and Gannon's Parallel XML
Processing algorithm~\cite{Lu:2007:PXP:1272457.1272462}. By using work-stealing queues,
the processing of an XML document can be easily load-balanced without knowing
its structure beforehand. For this test case, we serialize a model of an XML
document to a string. This is is done in a manner reminiscent of the quick sort
case. When a \texttt{Node} is examined by a \texttt{Worker}, all its children
are added to the queue. If the \texttt{Node} is a leaf, it is serialized, and
its parent \texttt{Node} is notified. Once all the children of a \texttt{Node}
have been serialized, the \texttt{Node} itself is serialized. The work is done
when the root \texttt{Node} has been serialized.

As with the Quick Sort case, our implementation of XML Serialization is not idempotent. This is caused by the mechanism through which a parent \texttt{Node} keeps track of its completed children: An atomic counter is incremented whenever a child \texttt{Node} has been processed. Therefore, if a child \texttt{Node} was processed more than once, the counter in the parent \texttt{Node} would not accurately represent the number of completed children. As a consequence, the parent \texttt{Node} would never be able to know whether \emph{all} of its children had been completed, or if the same children had simply been processed multiple times.

\subsection{Results}
Each of the four test cases were executed 50 times with the same random data. They were
tested with 8, 16, 32, and 48 threads. Due to technical difficulties, they were
not tested with more than 48 threads. See Section~\ref{sec:testing} for an
explanation. All benchmarks were performed on a machine running 
Ubuntu 12.04.4, with two AMD Opteron 8386 SE 16 core processors, and 128GB of
memory.

An overview of our results with 16 threads can be seen in Figure~\ref{fig:all_results}. The rest of our results can be found in Appendix~. \todo{Ref til appendix}

\begin{figure}
\begin{tikzpicture}
\begin{axis}[title={Results for 16 threads, normalized to ABP},  
  ylabel={Relative speedup},
  symbolic x coords={ABP, Chase-Lev, Chase-Lev (Shrinking), Idempotent (LIFO), Idempotent (FIFO),
                     Idempotent (Double-ended), Duplicating},
  xtick=data,
  x tick label style={rotate=315,anchor=west},
  legend style={at={(1.17,1)}, anchor=north,legend columns=1},
  legend cell align=left,
  ybar, bar width=4,
  height=200,
  width=300]
\addplot coordinates {
(ABP,1) (Chase-Lev, 1.065405) (Chase-Lev (Shrinking), 1.081225) (Idempotent (LIFO), 0)
(Idempotent (FIFO), 0) (Idempotent (Double-ended), 0) (Duplicating, 0)
};
\addplot coordinates {
(ABP,1) (Chase-Lev, 0.948556) (Chase-Lev (Shrinking), 0.949765) (Idempotent (LIFO), 0.770883)
(Idempotent (FIFO), 1.077969) (Idempotent (Double-ended), 0.803096) (Duplicating, 0.924355)
};
\addplot coordinates {
(ABP,1) (Chase-Lev, 1.411119) (Chase-Lev (Shrinking), 1.368644) (Idempotent (LIFO), 0)
(Idempotent (FIFO), 0) (Idempotent (Double-ended), 0) (Duplicating, 0)
};
\addplot coordinates {
(ABP,1) (Chase-Lev, 0.613657) (Chase-Lev (Shrinking), 0.89142) (Idempotent (LIFO), 0.184223)
(Idempotent (FIFO), 0.314387) (Idempotent (Double-ended), 0.109937) (Duplicating, 0.594886)
};
\legend{Quick sort, Spanning tree, XML, Raw}
\end{axis}
\end{tikzpicture}
\caption{Average of all results with 16 threads.}
\label{fig:all_results}
\end{figure}
