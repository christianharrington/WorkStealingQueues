%!TEX root = ../Work-stealing Queues.tex
\section{Benchmarks}
To test our different queue implementations, we developed a small framework for parallelizing computations using work-stealing queues. There are three main components, the \texttt{WorkerPool}, \texttt{Worker}, and \texttt{Node} traits. For each of the cases below, classes implemented these traits were created. The \texttt{WorkerPool} is in charge of starting the desired number of threads, and assigning each \texttt{Worker} to a thread. The \texttt{Worker} holds a queue, and contains the main logic for the case. \texttt{Node}s represent individual ``work units'', and are the elements contained in the queues.

\label{sec:benchmarks}
\subsection{Cases}
To test our different work-stealing queues, we have implemented four different examples.
\subsubsection{Raw Queue Operations} %Raw
The first, which we call ``Raw'', is meant to do as little work as possible, and thus to stress the queue implementations as much as possible. \texttt{RawTreeBuilder} builds a tree of \texttt{Node}s, which hold an arbitrary number of children, and nothing else. Each \texttt{Worker} simply looks at a \texttt{Node}, and adds all its children to the queue, before moving to the next \texttt{Node}. This means that queue operations will be the deciding factor for how long a run will take. It should be noted that this is an unrealistic workload, as no actual work is done, besides traversing the tree.

\subsubsection{Quick Sort} % Quick sort
Our second case is a simple quick sort implementation. It is simple in that it does not sort in-place. Instead, when splitting a string, each new child \texttt{Node} receives a full copy of the string to split. When a lower threshold for string length is reached, insertion sort is used. Once a string has been sorted using insertion sort, its parent \texttt{Node} is notified. If both the parent \texttt{Node}'s children have been sorted, the parent \texttt{Node} is added to the queue. When the parent \texttt{Node} is examined by a \texttt{Worker}, the two sorted substrings are combined, and the next parent \texttt{Node} in the hierarchy is notified. When both children of the root \texttt{Node} are combined, the entire string has been sorted.

\subsubsection{Spanning Tree} % Spanning tree
For our third case, we implemented the parallel spanning tree algorithm described by Bader and Cong~\cite{Bader04afast}, with some small differences. Their algorithm functions by first letting a single thread build a ``stub tree'', a small portion of the full tree built by randomly walking the graph. This tree's vertices are then evenly distributed into each queue. After this, each thread starts consuming work from their queues. Our implementation is slightly different, in that we do not build the initial stub tree. Instead, the first \texttt{Worker} is given a node in the graph to work on, and as it generates more work, the other \texttt{Worker}s can steal from it. This change was purely to make the implementation simpler.

\subsubsection{XML Serialization} % XML serialization
Our final case was inspired by Lu and Gannon's Parallel XML Processing~\cite{Lu:2007:PXP:1272457.1272462}. By using work-stealing queues, the processing of the XML document can be easily load-balanced without knowing its structure before hand. For this test case, we serialize a model of an XML document to a string. This is is done in a manner reminiscent of the quick sort case. When a \texttt{Node} is examined by a \texttt{Worker}, all its children are added to the queue. If the \texttt{Node} is a leaf, it is serialized, and its parent \texttt{Node} is notified. Once all the children of a \texttt{Node} have been serialized, the \texttt{Node} itself is serialized. The work is done when the root \texttt{Node} has been serialized.
