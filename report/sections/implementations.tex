%!TEX root = ../Work-stealing Queues.tex
\section{Implementations}
\label{sec:implementations}
\todo{More code examples}
In this section we will present the 6 queues we have implemented.
The first three are based on the Arora-Blumofe-Plaxton (ABP) queue, with later versions bringing various improvements.
The final three are so-called idempotent queues, which we will explain in Section~\ref{subsec:idempotent_queues}.

\subsection{Arora-Blumofe-Plaxton Queue}
The ABP queue\,\cite{Arora:1998:TSM:277651.277678} stores its tasks in a fixed size array, and uses \texttt{top} and \texttt{bottom} pointers to keep track of the beginning and end of the queue.
Pushing a task to the queue (\texttt{push}) is straightforward.
It is added to the array at the bottom, and \texttt{bottom} is incremented.
Taking tasks from the queue requires more work to ensure that the same work is not extracted twice, as \texttt{take} and multiple different \texttt{steal} operations can happen simultaneously.
Avoiding duplication is done by only allowing \texttt{steal} to return if the \texttt{top} has remained unchanged throughout the operation. 
This, however, is not sufficient\,\cite{Arora:1998:TSM:277651.277678}, as it is still susceptible to the ABA problem.
If the queue is emptied and refilled to the same size between the read from the queue (Figure~\ref{fig:abpsteal}, line 9) and the comparison with \texttt{top} (Figure~\ref{fig:abpsteal}, line 12) the read might no longer be valid.
To handle this, a \texttt{tag} value is added, which is incremented whenever the queue is completely emptied.
This \texttt{tag} together with \texttt{top} is now known as \texttt{age}.
Now, instead of just the \texttt{top} having to remain unchanged, the entire \texttt{age} must be the same.
While the ABP queue is relatively simple to implement, it suffers from its use of a fixed size array.
Because of this, one must know the maximum number of tasks beforehand, or risk overflow errors.

\begin{figure}
\begin{lstlisting}[language=scala,basicstyle=\ttfamily\bfseries\scriptsize,numbers=left]
final def steal(): Option[E] = {
    val oldAge = age.get
    val localBot = bottom

    if (localBot <= oldAge.top) { // Queue is Empty
      None
    }
    else {
      val v = queue.get(oldAge.top)
      val newAge = Age(oldAge.tag, oldAge.top + 1)

      if (age.compareAndSet(oldAge, newAge)) { 
      	// CAS for ABA prevention
        Some(v)
      }
      else {
        None
      }
    }
  }
\end{lstlisting}
\caption{ABP implementation of \texttt{steal}.}
\label{fig:abpsteal}
\end{figure}

\subsection{Chase-Lev Queue}
Chase and Lev designed an improved version of the ABP queue using circular arrays\,\citep{ChaseLev05}. 
The Chase-Lev implementation solves the fundamental problem from which the ABP queue suffers, namely the use of fixed-size arrays.
Besides the risk of overflows, the fixed-size arrays also lead to inefficient memory usage: In a system with \emph{n} threads and allocated memory \emph{m}, any queue in the system can hold at most $\frac{m}{n}$ elements.
Using dynamic circular arrays, any Chase-Lev queue can grow when necessary, which means that the shared memory does not have to be divided among the queues upfront.
To improve memory usage, Chase and Lev also describe a shrinking operation for the underlying circular arrays, such that memory is reclaimed whenever the actual size of an array is less than the allocated size by some constant factor.
Since the \texttt{top} pointer is never decremented, the size of a given Chase-Lev queue is only restricted to the size of \texttt{top} (Chase and Lev use a 64-bit integer).
Contrary to the ABP queue, the Chase-Lev implementation does not need a tag to avoid the ABA problem.
As a direct consequence of the \texttt{top} pointer never decrementing, any thread will always notice if the contents of the queue have changed (using a \emph{compare-and-swap} operation on \texttt{top}).
For this project we have implemented two versions of the Chase-Lev queue, one with shrinking and one without.
While all our tests have been run in a garbage-collected environment on the JVM, the Chase-Lev queue does not rely on garbage collection for memory management.

\subsection{Idempotent Queues}
\label{subsec:idempotent_queues}
The previously discussed queues guarantee that no task will be extracted twice. 
This can be relaxed in some cases to possibly improve performance\,\cite{Michael:2009:IWS:1594835.1504186}. 
The rest of the queues do not provide this guarantee, which means that they can only be used in cases where the task is idempotent, i.e.\ they can be performed multiple time without changing the result.

We have implemented three different approaches to idempotent work stealing\,\cite{Michael:2009:IWS:1594835.1504186}.
Generally, all three work in a similar fashion, but the order in which tasks are extracted differ from queue to queue.
All three queues keep their tasks in an array, but use different ways of keeping track of the ends of the queue
When talking about these structures we use the term ``queue'' loosely to mean a data structure that keeps its elements in some order.

\paragraph{The Idempotent LIFO Queue}\,\cite{Michael:2009:IWS:1594835.1504186} does not need a special pointer to track the beginning of the queue, as tasks are always added to, and extracted from the end.
The end of the queue is determined by a \texttt{tail} pointer, which always points to the last task in the queue.
To prevent the ABA problem, the \texttt{tail} is bundled with a \texttt{tag} in an \texttt{anchor}.
This \texttt{anchor} must remain unchanged between the read of a stolen element and the return of this element.
It is worth noting that, in contrast to the other queues, this queue (along with the Idempotent FIFO queue) \texttt{steal} and \texttt{take} from the same end of the queue.

\paragraph{The Idempotent FIFO Queue}\,\cite{Michael:2009:IWS:1594835.1504186} adds tasks to the end of the queue, defined by a \texttt{tail} pointer, and removes from the beginning of the queue, determined by a \texttt{head} pointer.
This means both \texttt{take} and \texttt{steal} operate on the same end of the queue.
Interestingly, this implementation does not need ABA prevention at all.
This is because \texttt{head} can only decrease in one case:
\begin{enumerate}
  \item A \texttt{take} operation begins, and reads the \texttt{head} pointer.
  \item Before the \texttt{take} completes, one or more threads \texttt{steal}, incrementing \texttt{head}.
  \item The original \texttt{take} operation completes, incrementing \texttt{head} by 1 compared to its original value, the result of which is lower than the \texttt{head} set by the other threads.
\end{enumerate}
However, this does not matter, as a decreasing head would only cause a problem if executed concurrently with a \texttt{push} operation.
This is not possible, as only the thread owning the queue can perform the \texttt{take} and \texttt{push} operations.

\paragraph{The Idempotent Double-Ended Queue}\,\cite{Michael:2009:IWS:1594835.1504186} is unusual, as elements are pushed to and stolen from the beginning of the queue, while they are taken from the end. 
This means that this implementation also needs ABA prevention.
This is done in a similar way as the LIFO queue, with a \texttt{anchor} packed with the \texttt{head}, \texttt{size}, and a \texttt{tag}.

\subsection{Duplicating Queue}
Similar to the idempotent queues, the duplicating queue\,\cite{Leijen:2009:DTP:1639949.1640106} can potentially return a pushed task more than once. 
When stealing from this queue, the stolen task is either removed from the queue or just duplicated. 
When a task is the only element in the queue, concurrent \texttt{steal} (stealing from the \texttt{head}) and \texttt{take} (taking from the \texttt{tail}) operations will access the same task.
In this case we say that the task has been ``duplicated''. 
Two threads cannot steal the same element from the queue as stealing requires a lock.

Furthermore, if a new element is pushed just after the above duplication, the \texttt{head} and \texttt{tail} pointers are inconsistent\,\cite{Leijen:2009:DTP:1639949.1640106}.
This is solved by another pointer \texttt{tailMin} indicating the minimal index at which a task has been taken. When checking if the queue is empty when taking, \texttt{head} is compared to the smallest value of \texttt{tailMin} and \texttt{tail}.

\subsection{Comparison}
Our implementations can be split into two categories: those that are idempotent, and those that are not. The idea behind the idempotent queues is that the queue operations themselves are simpler, but at the risk of performing the same task multiple times. This means that we can expect the idempotent queues to perform well in cases where there are many queue operations. They will, however, not work in cases where the same work is not allowed to be done twice.

Amongst the idempotent queues, the duplicating queue stands out. 
Since it is only idempotent when the owner and exactly one thief attempts to access a queue with exactly one element, duplication of work happens less often.
This comes at the cost of slowing down concurrent \texttt{steal} operations, as the lock only lets one thread steal at a time.

The ABP-based queues are very similar, but the key difference is that the size of the ABP queue's array is fixed, whereas Chase-Lev can grow.
We have also made an implementation of Chase-Lev that also shrinks the queue, which should be even more memory efficient, at the cost of performance.
This means that while ABP is less memory efficient, it might perform better than the two Chase-Lev queues, as it does not need to grow or shrink the queue. 
