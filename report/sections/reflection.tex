%!TEX root = ../Work-stealing Queues.tex
\section{Reflection}
\label{sec:Reflection}
Besides the trouble with testing described in Section~\ref{sec:testing}, we
encountered several problems when trying to perform benchmarks on our various
queue implementations. The most important of these issues are described below.
 
\subsection{Cases For Idempotent Queues}
\label{sub:cases_for_idempotent_queues}
While performing the benchmarks, we realized we had made a mistake when
choosing our cases. First of all, when choosing which cases to implement, we
did not consider which could easily be run with the idempotent queues. This
meant we only had two cases to test the idempotent queues with. Further, the
two cases that happen to work with the idempotent queues are the two cases
that involve the least work for each thread. This makes it hard for us to tell
to which degree the performance of the idempotent queues is effected by queue
operations versus the overhead of possibly performing a piece of work multiple
times.

\subsection{Statistics For Queue Operations}
\label{sub:statistic_for_queue_operations}
Perhaps the greatest issue we found when analyzing our benchmarks, was the lack
of detailed metrics for the queues. For instance, it would be very useful to be
able to measure how many unsuccessful steals occur, and how long each of these
operations take. This degree of detailed information would let us reason more
accurately about the performance characteristics of our different queues and
cases. Unfortunately it is difficult to make these types of measurements
without affecting the execution of the program. If we were only working with
single threaded workloads, the additional overhead from the logging would be
easy to account for. But in our program, where specific interleavings of
multiple threads can produce reasonably large changes in execution time, it is
hard to account for the effects of logging. Regrettably, we did not find an
acceptable solution within the scope of this project.

\subsection{Hardware}
\label{sub:hardware}
One unexpected area where we encountered difficulties was with the processors
on the machine used for benchmarks. For most of the project we were under the
impression that it was a 16 core Intel Xeon processor with hyper-threading.
Later, we found out it had two AMD Opteron processors with 16 cores each.
Strangely, the information in \texttt{/proc/cpuinfo} and the output from
\texttt{lscpu} both implied only 8 physical cores were available on each
processor, with each core exposing two logical cores. To our knowledge, AMD
does not support a technology equivalent to hyper-threading. Because of this,
some uncertainty exists as two how the machine is actually configured.
