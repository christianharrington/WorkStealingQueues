%!TEX root = ../Work-stealing Queues.tex
\section{Introduction}
\label{sec:introduction}
A popular way of scheduling parallel workloads across multiple threads is through the use of work-stealing queues. 
Several variations of the canonical work-stealing queue exist, featuring different advantages and drawbacks.
In this report we describe six approaches, in order present an overview of possible strategies for implementing work-stealing queues.
We have also implemented these queues for the Java Virtual Machine (JVM), using the Scala programming language, along with several cases that use these queues.
This is done in order to compare the performance of the different queues in different situations. 

This report is structured as follows: In Section~\ref{sec:background}, we cover
the necessary background, including different approaches to concurrency, as well
as work-stealing queues themselves. We
describe the six different queues we have implemented, along with their
differences in Section~\ref{sec:implementations}. Section~\ref{sec:benchmarks} describes the queue implementations, along with our benchmarking results for each case. An analysis of
these results are presented in Section~\ref{sec:performance_analysis}. Our
testing approach is explained in Section~\ref{sec:testing}, while we reflect
on the project in Section~\ref{sec:Reflection}. We conclude in 
Section~\ref{sec:conclusion}.
