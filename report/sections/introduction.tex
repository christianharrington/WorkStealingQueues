%!TEX root = ../Work-stealing Queues.tex
\section{Introduction}
\label{sec:introduction}
A popular way of scheduling parallel workloads across multiple threads is
through use of the work-stealing queue. Several different variations of the
work-stealing queue exists, featuring different advantages and drawbacks. In
this report we describe 6 different queue varieties, in order to present an
overview of the different approaches available when implementing a
work-stealing queue. We have also implemented these queues in the Scala
programming language, along with several different cases that use them. This is
done in order to compare the performance of the different queues in different
situations. 

This report is structured as follows: In Section~\ref{sec:background}, we cover
the necessary background, covering different approaches to concurrency, as well
as work-stealing queues themselves. In Section~\ref{sec:implementations} we
describe the six different queues we have implemented, along with their
differences. Section~\ref{sec:benchmarks} describes the different cases we have
implemented, along with our benchmarking results for each case. An analysis of
these results are presented in Section~\ref{sec:performance_analysis}. Our
testing approach is explained in Section~\ref{sec:testing}, while we reflect
on the project in Section~\ref{sec:Reflection}. We conclude in 
Section~\ref{sec:conclusion}.
